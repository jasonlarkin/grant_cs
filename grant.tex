
\section{Introduction}\label{Introduction}

\subsection{Two Computational Domains: Local and Remote}

Development and testing are performed on the computer sitting right in front of you: your laptop, desktop, or 
even tablet and phone.

\subsubsection{Two Computational Platforms: Linux and Everthing Else }
80 $/%$ of the worlds supercomputers use Linux. So why would you chose a Local machine that runs anything else?

There are many reasons why someone might choose a different operating system as their computational 
platform.  

Windows: a historical lion's share of the PC OS market, which has been dwindling significantly 
due to recent decisions regarding the fusion of the Windows OS with tablet and phone devices.

Mac OS: has made . Mac OS is based on the unix kernel, making it an acceptable alternative to Linux OS.  However, 
notable differences such as... . The sacrifices in consistency between Local and Remote systems is not 
worth the trouble from curation and current usage 

None of the reasons


\subsection{New Population}

http://www.universityworldnews.com/article.php?story=20130403121244660

The new population needs to be introduced to computational tools quickly and efficiently.  
The new population also needs to learn the fundamentals of computation. The right balance 
must be struck between tools that have been abstraced away from the .





\section{Research Curation}\label{}

\subsection{Source, Data, and System Curation}\label{}

Source may be the most important result of a computaional research project. It should provide 
future researchers to <a href="http://en.wikipedia.org/wiki/Bootstrapping">"bootstrap"</a>. 

We need new protocols for the way in which data is curated.

Data storage capacties are moving targets, they enjoy the same kind of exponential 
growth seen in other metrics of computational science performance, such as CPU...

An increasing data heirachy might look something like this:

1 kb = a key funciton or kernel in the project's source code.
100 kb = a plot/figure 
1000 kb (1 Mb) = an entire project's source code
1,000,000 kb (1 Gb) = an entire project's data storage

With increases in data storage capacitis and transfer rates across the 
entire computational spectrum (local, wireless, etc), 
we can expect plot/figure sizes and possibly an entire project's data storage to scale with 
these global increases in transfer rates. 



\subsection{Data Curation}\label{}


A Virtual Machine can be made immutable . There can always be an original copy 
of the machine's exact state of hardware and software.

A Virtual Machine is usable by anyone who has been given secure access, anywhere there 
is an internet connection. (this is pretty much everywhere, and growing exponentially).

A Virtual Development Envinronment allows anyone with secure access 
.

Critical design aspects can be communicated instantly.  

Can reduce 

Case Study : how many emails about a 1-2 line code fix.
Distribution of error types by: lines_of_code


Cloud service hardware is some of the most secure computing systems in the world. 
Fortune 500 companies (Google, Amazon, etc) invest billions of $ into the design 
of their advanced computing systems. 

Case Study: maintenance of beowulf cluster (96 cpu):


Deliverable: a highly-configured Virtual Machine instance for scientific computing.  
A design for a scalable-instance paradigm 

\section{Case Studies}

